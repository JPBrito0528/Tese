\prefacesection{Acknowledgements}



\prefacesection{Abstract}

\noindent WTVision, a business that specializes in real-time graphics and augmented reality solutions for the broadcast and entertainment industries, provided a corporate setting for this internship. With the use of WTVision's technology, immersive experiences that demand a high degree of visual accuracy and precision are produced. Accurate recreation of a camera lens distortion in a virtual environment is a major problem in these settings and an accurante determinatation of distortion coeffients is mandatory for recreating immersive AR experiences.

\noindent It's important to clarify that in augmented and virtual reality applications, where seamless alignment between virtual features and real-world visuals is necessary to deliver an immersive user experience, the aim is not to minimize the camera's distortion effect. Instead, the goal is to make the virtual world look geometrically aligned to the real world does. However, current calibration methods are extremely time-consuming and require meticulous manual adjustments at different zoom and focus levels. Manual calibration typically takes between 4 to 8 hours and is heavily reliant on human factors, increasing the likelihood of errors.

\noindent The objective of this project is to create a lens calibration algorithm that optimizes and automates the procedure, cutting the overall calibration time down to about 30 minutes while maintaining accuracy. The suggested approach makes better use of cutting-edge methods like image analysis and machine learning to fix distortion issues. The study encompasses a thorough examination of current calibration techniques, the creation and modeling of a novel algorithm, and exhaustive experimentation on various types of lens under different circunstances.

\noindent The study's findings should greatly enhance the workflow for calibration teams by enabling faster, more accurate calibrations with less need for human intervention. The suggested method could significantly increase productivity in fields where accurate camera lens calibration is essential, especially for augmented and virtual reality applications.

\prefacesection{Resumo}
\noindent A WTVision é uma empresa especializada em soluções de gráficos em tempo real e realidade aumentada para a indústria de transmissão e entretenimento. A tecnologia da WTVision permite criar experiências imersivas que exigem um alto grau de precisão e exatidão visual. Um dos principais desafios nesses cenários é a recriação precisa da distorção de lentes de câmera em um ambiente virtual, sendo essencial a determinação acurada dos coeficientes de distorção para a criação de experiências imersivas de realidade aumentada.

\noindent É importante esclarecer que, em aplicações de realidade aumentada e virtual, onde o alinhamento perfeito entre elementos virtuais e imagens do mundo real é essencial para proporcionar uma experiência imersiva, o objetivo não é minimizar a distorção da câmera. Em vez disso, busca-se garantir que o mundo virtual esteja geometricamente alinhado com o mundo real. No entanto, os métodos atuais de calibração são extremamente demorados e exigem ajustes manuais meticulosos em diferentes níveis de zoom e foco. A calibração manual geralmente leva entre 4 a 8 horas e depende fortemente de fatores humanos, aumentando a probabilidade de erros.

\noindent O objetivo deste projeto é desenvolver um algoritmo de calibração de lentes que otimize e automatize o procedimento, reduzindo o tempo total de calibração para cerca de 30 minutos, sem comprometer a precisão. A abordagem proposta faz uso de técnicas avançadas, como análise de imagens e aprendizado de máquina, para corrigir problemas de distorção. O estudo abrange uma análise detalhada dos métodos atuais de calibração, o desenvolvimento e modelagem de um novo algoritmo e experimentações extensivas com diferentes tipos de lentes sob diversas condições.

\noindent Os resultados deste estudo devem melhorar significativamente o fluxo de trabalho das equipes de calibração, permitindo calibrações mais rápidas e precisas, com menor necessidade de intervenção humana. O método proposto pode aumentar substancialmente a produtividade em áreas onde a calibração precisa de lentes de câmera é essencial, especialmente em aplicações de realidade aumentada e virtual.

