\chapter{Image Deblurring NN} \label{chap:Time}

In real world applications, the images are often blurred due to, mainly, out-of-focus camera.
For the calibration algorithm, a minimum image quality is required to estimate the camera parameters.
To improve the image quality, we can use a deblurring algorithm. In this chapter, we will discuss the deblurring algorithm and 
its implementation using a neural network.

\section{Dataset}

The dataset used for training the neural network is a set of blurred images and their corresponding sharp images. 
These images were obtained using blender, since its able to recreate accurate images of an out-of-focus camera.
Using a python script generated xxxx set of blurred and the corresponding sharp images. Each image consists of a red chessboard square 
with random size, position and aspect ratio with a random studio in the background.

\section{Deblurring using EDSR}

For the deblurring task, we used the Enhanced Deep Super-Resolution (EDSR) network, originally designed for single-image 
super-resolution. Although EDSR was developed to upscale low-resolution images to high-resolution outputs, it has been shown to 
be effective for image restoration tasks such as deblurring.

EDSR is a deep convolutional neural network that removes unnecessary modules, such as batch normalization layers, from its 
predecessor models (e.g., SRResNet) to improve performance. It consists of multiple residual blocks without downsampling, enabling the network 
to maintain spatial information and focus on learning fine details.

In our case, the input to the EDSR network is a blurred image, and the output is its deblurred (sharpened) version. The network is trained to 
minimize the pixel-wise loss between the predicted deblurred image and the ground truth sharp image.

By using EDSR, we benefit from a powerful architecture capable of restoring image details that were lost due to blur, thus enabling better 
performance in downstream tasks such as camera calibration.

\subsection{Loss Function}

Since the images used are very blured, simply using EDSR was not be enough. To tackle this problem,
the algorithm is trained to focus on debluring the chessboard square, which is the only area of interest in the image, giving
little value to the rest of the image. 

This is done by recording the corners coordinates of the chessboard square in the training dataset. By applying a crop on the image blured and 
sharp images. By using the MSELoss function, the algorithm will learn to deblur the chessboard square. In order to not ignore completely
the rest of the image, the algorithm also learns to deblur the rest of the image, but with a lower weight. This is done
by multiplying the MSELoss score of the chessboard square 0.9, and the rest of the image by 0.1 and then adding both.

*The loss functions is not converging this way, alternative needed*



